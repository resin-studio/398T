\begin{frame}
  \frametitle{Proof by contrapositive}

  \begin{itemize}
  \item Sometimes a direct proof of $P\rightarrow Q$ can be really
    annoying
  \item Proving the \textbf{contrapositive} $\neg Q\rightarrow \neg P$
    might be easier
  \item We can always opt to do this because
    \begin{mathpar}
      (P\rightarrow Q)\equiv (\neg Q \longrightarrow \neg P)
    \end{mathpar}
  \end{itemize}

  \exer What is the contrapositive of these claims?
  \begin{itemize}
  \item If today is Tuesday, then we have class today.
  \item If this is a cat, then it does not bark.
  \end{itemize}
  
\end{frame}

\begin{frame}
  \frametitle{Proof by contrapositive: Example}

  \begin{prop}
    If $r$ is an irrational number, then $\sqrt{r}$ is also an
    irrational number.
  \end{prop}
  \begin{proof}
    Prove the contrapositive: if $\sqrt{r}$ is not an irrational
    number, then $r$ is not an irrational number.\\

    We assume that $\sqrt{r}$ is a rational number. Then, there exists
    some integers $m$ and $n$ such that $\sqrt{r} =
    \frac{m}{n}$.\\

    Square both sides to get $\sqrt{r}^2 = \frac{m^2}{n^2}$. We know
    that a square of integer is an integer, so $m^2$ and $n^2$ are
    both integers. We get that $r = \frac{m^2}{n^2}$, which means that
    $r$ is a rational number and not an irrational number.
  \end{proof}
  \qquad\\ \refl How might you prove this directly? Is it easier or
  harder to prove this directly or with contrapositive?

\end{frame}
