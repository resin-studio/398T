\begin{frame}
  \frametitle{Proof by cases}
  \begin{itemize}
  \item A form of direct proof in which
    \begin{enumerate}
    \item The hypothesis is decomposed into distinct cases and
    \item The conclusion is demonstrated separately in each case
    \end{enumerate}
    \qquad\\\qquad\\
  \item Valid case decompositions are \textit{exhaustive}: they
    articulate all and only the possibilities admitted by the original
    hypothesis
    \begin{itemize}
    \item Very valid: $x\in\mathbb{N}\rightarrow x\text{ is even}\lor x\text{ is odd}$
    \item Very invalid: $x\in\mathbb{N}\rightarrow x=0\lor x=1\lor x=3$
    \end{itemize}
    \qquad\\\qquad\\
  \item Sometimes called ``proof by exhaustion''
  \end{itemize}
\end{frame}


\begin{frame}
  \frametitle{Proof by cases: Example}
  \begin{prop}
    For all integers $n$, $n^2+n$ is even.
  \end{prop}
  \begin{proof}
    By cases on the evenness of $n$.

    If $n$ is even, then $n^2$ is even and hence $n^2 +n$ is even.

    If $n$ is odd, then $n^2$ is odd and hence $n^2 +n$ is even.
  \end{proof}
  \qquad\\
  \refl Think of how you might prove this proposition without
  resorting to cases
\end{frame}

\begin{frame}
  \frametitle{Proof by cases: Inference rules}
  \begin{mathpar}
    \inferrule
        {
          (\phi_1\lor\cdots\lor\phi_n)\rightarrow \psi
        }
        {
          \phi_1\rightarrow \psi\\
          \cdots\\
          \phi_n\rightarrow \psi
        }
        \textsc{\ \ CaseIntro}

        \inferrule
            {
              \phi_1\rightarrow \psi\\
              \cdots\\
              \phi_n\rightarrow \psi
            }
            {
              (\phi_1\lor\cdots\lor\phi_n) \rightarrow \psi
            }
            \textsc{\ \ CaseElim}
  \end{mathpar}
  In short, $((\phi_1\lor\cdots\lor\phi_n) \rightarrow
  \psi)\equiv((\phi_1\rightarrow\psi)\land\cdots\land(\phi_n\rightarrow\psi))$.
\end{frame}

\begin{frame}
  \frametitle{Proof by cases}
  \begin{itemize}
  \item Pros
    \begin{itemize}
    \item Conceptually simple
    \item Well-chosen cases can make proofs easier to write
    \end{itemize}
    \qquad\\\qquad\\
  \item Cons
    \begin{itemize}
    \item Too many cases are fun for no one (especially readers)
    \item Considered inelegant
    \end{itemize}
    \qquad\\\qquad\\
  \end{itemize}
  \refl Come up with your own proposition and prove it by cases
\end{frame}
