\begin{frame}
  \frametitle{Basic inference: introduction}
  
  
  \begin{itemize}
    \item An inference rule is a logical step of reasoning
    \item An inference rule says "from P, we can deduce Q" 
    \item Or in other words: "Q, under the assumption of P" 
    \item Some common notations 
    \begin{mathpar}
      P \vdash Q

      \inferrule{P}{Q} 

      P \Longrightarrow Q
    \end{mathpar}

    \item from \(9 + x^2 = 25\), we can deduce \(b = 4\)
    \begin{mathpar}
      9 + x^2 = 25 \vdash |x| = 4

      \inferrule 
        {9 + x^2 = 25} 
        {|x| = 4}

      9 + x^2 = 25 \Longrightarrow |x| = 4
    \end{mathpar}
  \end{itemize}


\end{frame}

\begin{frame}
  \frametitle{Basic inference: proof}
  
  
  \begin{itemize}
    \item A proof is a combination of inference rules  
    \item A proof of \(P \vdash Q\), i.e. "Q, under the assumption of P": 
    \begin{mathpar}
      \inferrule
        {
          \inferrule
            { }
            {R_1}
          \\
          \inferrule
            {P}
            {R_2}
        }
        {Q} 
    \end{mathpar}
  \end{itemize}


\end{frame}

\begin{frame}
  \frametitle{Basic inference: implication and negation}
  
  \begin{itemize}
    \item Implication: \(P \rightarrow Q\), i.e. "if P then Q"
  \end{itemize}
  \begin{mathpar}
    \\
    \inferrule
      {
        P
        \\\\
        \vdots
        \\\\
        Q
      } 
      {
        P \rightarrow Q
      } 
    \textsc{\ \ ImpIntro}

    \inferrule
      {
        P \rightarrow Q
        \\
        P
      } 
      {
        Q
      } 
    \textsc{\ \ ImpElim}
    \\
  \end{mathpar}

  \begin{itemize}
    \item Negation: \(\neg P\), i.e. "not P"
  \end{itemize}
  \begin{mathpar}
    \\
    \inferrule
      {
        P \rightarrow Q
        \\
        P \rightarrow \neg Q
      } 
      {
        \neg P
      } 
    \textsc{\ \ NegIntro}

    \inferrule
      {
        \neg P
      } 
      {
        P \rightarrow Q
      } 
    \textsc{\ \ NegElim}
    \\
\end{mathpar}
\end{frame}


\begin{frame}
  \frametitle{Basic inference: conjunction and disjunction}

  \begin{itemize}
    \item Conjunction: \(P \wedge Q\), i.e. "P and Q"
  \end{itemize}
  \begin{mathpar}
    \\
    \inferrule
      {
        P \\ Q
      } 
      {
        P \wedge Q
      } 
    \textsc{\ \ ConjIntro}

    \inferrule
      {
        P \wedge Q
      } 
      {
        P
      } 

    \inferrule
      {
        P \wedge Q
      } 
      {
        Q
      } 
    \textsc{\ \ ConjElim}
    \\
  \end{mathpar}

  \begin{itemize}
    \item Disjunction: \(P \vee Q\), i.e. "P or Q"
  \end{itemize}
  \begin{mathpar}
    \\
    \inferrule
      {
        P 
      } 
      {
        P \vee Q
      } 

    \inferrule
      {
        Q 
      } 
      {
        P \vee Q
      } 
    \textsc{\ \ DisjIntro}

    \inferrule
      {
        P \vee Q
        \\\\
        P \rightarrow R 
        \\
        Q \rightarrow R 
      } 
      {
        R 
      } 
    \textsc{\ \ DisjElim}
    \\
  \end{mathpar}

\end{frame}



\begin{frame}
  \frametitle{Basic inference: universal and existential}

  \begin{itemize}
    \item Universal: \(\forall x . P(x)\), i.e. "for all x, P(x)"
  \end{itemize}
  \begin{mathpar}
    \\
    \inferrule
      {
        \text{fresh}\ y
        \\
        P(y)
      } 
      {
        \forall x . P(x)
      } 
    \textsc{\ \ UnivIntro}

    \inferrule
      {
        \forall x . P(x)
      } 
      { 
        P(t) 
      } 
    \textsc{\ \ UnivElim}
    \\
  \end{mathpar}

  \begin{itemize}
    \item Existential: \(\exists x . P(x)\), i.e. "there's an x, such that P(x)"
  \end{itemize}
  \begin{mathpar}
    \\
    \inferrule
      {
        P(t)
      } 
      {
        \exists x . P(x)
      } 
    \textsc{\ \ ExisIntro}

    \inferrule
      {
        \exists x . P(x)
        \\
        \text{fresh}\ y
      } 
      { 
        P(y) 
      } 
    \textsc{\ \ ExisElim}
    \\
  \end{mathpar}

\end{frame}

\begin{frame}
  \frametitle{Basic inference: exercises}

  Match each phrase with one or more inference rules:
  \begin{enumerate}
    \item Life is beautiful, so either life is beautiful or pigs fly.
    \item Obviously pigs don't fly, so if pigs fly then I'll give you a bitcoin for a pickle. 
    \item Given that they're a PhD student, you can deduce that they are tired and stressed.     
    Thus if they're a PhD student, then they're tired and stressed.
    \item Austin is a city in Texas, so there's a city in Texas.
    \item All texans like BBQ and some texans are vegan, so even vegans like BBQ.
    \item There's a dog in the yard; let's call it Fido; so Fido is in the yard. 
  \end{enumerate}
\end{frame}

