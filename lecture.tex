\documentclass{beamer}

\usepackage{mathpartir}
\usepackage{hyperref}
\usepackage{mathtools}
\usepackage{amsthm}

\newcommand{\refl}{\textcolor{blue}{Reflection:~}}
\newcommand{\exer}{\textcolor{blue}{Exercise:~}}
\newtheorem*{prop}{Proposition}

\title{Proof Strategies}
\author{Group 9 From Outer Space}
\date{17 November 2022}

\begin{document}

\begin{frame}
  \titlepage
\end{frame}

\begin{frame}
  \frametitle{Overview}
  \begin{itemize}
  \item Introduction\\
  \item Basic inference\\
  \item Proof strategies
    \begin{itemize}
    \item Direct proof
    \item Proof by cases
    \item Proof by contrapositive\\
    \end{itemize}
  \end{itemize}
\end{frame}

\begin{frame}
  \frametitle{Introduction}
  \begin{itemize}
  \item A mathematical proof is a sound argument that a certain
    conclusion logically follows from stated hypotheses\\
  \item Proofs enable us to synthesize new truths from old ones\\
  \item Proofs can be more or less formal, depending on the intended
    audience
    \begin{itemize}
    \item Less formal: human readers
    \item More formal: machine readers\\
    \end{itemize}
  \item \textcolor{blue}{This lecture:} Some basic strategies for
    structuring correct informal proofs
  \end{itemize}
\end{frame}

\begin{frame}
  \frametitle{Basic inference: introduction}
  
  
  \begin{itemize}
    \item An inference rule is a logical step of reasoning
    \item An inference rule says "from P, we can deduce Q" 
    \item Or in other words: "Q, under the assumption of P" 
    \item Some common notations 
    \begin{mathpar}
      P \vdash Q

      \inferrule{P}{Q} 

      P \Longrightarrow Q
    \end{mathpar}


    \item example: "from \(9 + x^2 = 25\), we can deduce \(x = 4\)"
    \begin{mathpar}
      \inferrule 
        {9 + x^2 = 25} 
        {|x| = 4}
    \end{mathpar}
  \end{itemize}


\end{frame}

\begin{frame}
  \frametitle{Basic inference: proof}
  
  
  \begin{itemize}
    \item A proof is a combination of inference rules  
    \item A proof of \(P \vdash Q\), i.e. "Q, under the assumption of P": 
    \begin{mathpar}
      \inferrule
        {
          \inferrule
            { }
            {R_1}
          \\
          \inferrule
            {P}
            {R_2}
        }
        {Q} 
    \end{mathpar}
  \end{itemize}


\end{frame}

\begin{frame}
  \frametitle{Basic inference: implication and negation}
  
  \begin{itemize}
    \item Implication: \(P \rightarrow Q\), i.e. "if P then Q"
  \end{itemize}
  \begin{mathpar}
    \\
    \inferrule
      {
        P
        \\\\
        \vdots
        \\\\
        Q
      } 
      {
        P \rightarrow Q
      } 
    \textsc{\ \ ImpIntro}

    \inferrule
      {
        P \rightarrow Q
        \\
        P
      } 
      {
        Q
      } 
    \textsc{\ \ ImpElim}
    \\
  \end{mathpar}

  \begin{itemize}
    \item Negation: \(\neg P\), i.e. "not P"
  \end{itemize}
  \begin{mathpar}
    \\
    \inferrule
      {
        P \rightarrow Q
        \\
        P \rightarrow \neg Q
      } 
      {
        \neg P
      } 
    \textsc{\ \ NegIntro}

    \inferrule
      {
        \neg P
      } 
      {
        P \rightarrow Q
      } 
    \textsc{\ \ NegElim}
    \\
\end{mathpar}
\end{frame}


\begin{frame}
  \frametitle{Basic inference: conjunction and disjunction}

  \begin{itemize}
    \item Conjunction: \(P \wedge Q\), i.e. "P and Q"
  \end{itemize}
  \begin{mathpar}
    \\
    \inferrule
      {
        P \\ Q
      } 
      {
        P \wedge Q
      } 
    \textsc{\ \ ConjIntro}

    \inferrule
      {
        P \wedge Q
      } 
      {
        P
      } 

    \inferrule
      {
        P \wedge Q
      } 
      {
        Q
      } 
    \textsc{\ \ ConjElim}
    \\
  \end{mathpar}

  \begin{itemize}
    \item Disjunction: \(P \vee Q\), i.e. "P or Q"
  \end{itemize}
  \begin{mathpar}
    \\
    \inferrule
      {
        P 
      } 
      {
        P \vee Q
      } 

    \inferrule
      {
        Q 
      } 
      {
        P \vee Q
      } 
    \textsc{\ \ DisjIntro}

    \inferrule
      {
        P \vee Q
        \\\\
        P \rightarrow R 
        \\
        Q \rightarrow R 
      } 
      {
        R 
      } 
    \textsc{\ \ DisjElim}
    \\
  \end{mathpar}

\end{frame}



\begin{frame}
  \frametitle{Basic inference: universal and existential}

  \begin{itemize}
    \item Universal: \(\forall x . P(x)\), i.e. "for all x, P(x)"
  \end{itemize}
  \begin{mathpar}
    \\
    \inferrule
      {
        \text{fresh}\ y
        \\
        P(y)
      } 
      {
        \forall x . P(x)
      } 
    \textsc{\ \ UnivIntro}

    \inferrule
      {
        \forall x . P(x)
      } 
      { 
        P(t) 
      } 
    \textsc{\ \ UnivElim}
    \\
  \end{mathpar}

  \begin{itemize}
    \item Existential: \(\exists x . P(x)\), i.e. "there's an x, such that P(x)"
  \end{itemize}
  \begin{mathpar}
    \\
    \inferrule
      {
        P(t)
      } 
      {
        \exists x . P(x)
      } 
    \textsc{\ \ ExisIntro}

    \inferrule
      {
        \exists x . P(x)
        \\
        \text{fresh}\ y
      } 
      { 
        P(y) 
      } 
    \textsc{\ \ ExisElim}
    \\
  \end{mathpar}

\end{frame}

\begin{frame}
  \frametitle{Basic inference: exercises}

  Match each phrase with one or more inference rules:
  \begin{enumerate}
    \item Life is beautiful, so either life is beautiful or pigs fly.
    \item Obviously pigs don't fly, so if pigs fly then I'll give you a bitcoin for a pickle. 
    \item Given that they're a PhD student, you can deduce that they are tired and stressed.     
    Thus if they're a PhD student, then they're tired and stressed.
    \item Austin is a city in Texas, so there's a city in Texas.
    \item All texans like BBQ and some texans are vegan, so even vegans like BBQ.
    \item There's a dog in the yard; let's call it Fido; so Fido is in the yard. 
  \end{enumerate}
\end{frame}


\begin{frame}
    \frametitle{Direct Proof}
  
    \begin{itemize}
    \item We need to prove a proposition of form $P \longrightarrow Q$.
    \item To do that we assume $P$ and use rules of inference and other relevant facts to show $Q$.
    \item Let's look at an example.
    \end{itemize}
  
  \end{frame}
  
  \begin{frame}
    \frametitle{Direct Proof: Example}
  
    \begin{prop}
      If $n$ is even, then $n^2$ is also even.
    \end{prop}
    \begin{proof}
      Assume $n = 2k$ for some number $k$, i.e. $n$ is even. Then $n^2 = (2k)^2 = 4 k^2 = 2 \cdot 2k^2$. This proves $n^2$ is also even.
    \end{proof}
  
  \end{frame}
  
  
  \begin{frame}
    \frametitle{Homework problems:}
  
   \begin{prop}
      If $n$ is a natural number, prove that $n^2$ is either divisible by $3$ or leaves a remainder of $1$ when divided by $3$.
    \end{prop}
   
     \begin{prop}
      If $p$ is a prime number greater than $3$, prove that $p$ leaves a remainder of $1$ or $5$ when divided by $6$.
    \end{prop}
  
  \end{frame}
\begin{frame}
  \frametitle{Proof by cases}
  \begin{itemize}
  \item A form of direct proof in which
    \begin{enumerate}
    \item The hypothesis is decomposed into distinct cases and
    \item The conclusion is demonstrated separately in each case
    \end{enumerate}
    \qquad\\\qquad\\
  \item Valid case decompositions are \textit{exhaustive}: they
    articulate all and only the possibilities admitted by the original
    hypothesis
    \begin{itemize}
    \item Very valid: $x\in\mathbb{N}\rightarrow x\text{ is even}\lor x\text{ is odd}$
    \item Very invalid: $x\in\mathbb{N}\rightarrow x=0\lor x=1\lor x=3$
    \end{itemize}
    \qquad\\\qquad\\
  \item Sometimes called ``proof by exhaustion''
  \end{itemize}
\end{frame}


\begin{frame}
  \frametitle{Proof by cases: Example}
  \begin{prop}
    For all integers $n$, $n^2+n$ is even.
  \end{prop}
  \begin{proof}
    By cases on the evenness of $n$.

    If $n$ is even, then $n^2$ is also even and hence so is $n^2 +n$.

    If $n$ is not even, then $n$ is odd.  This means $n^2$ is also odd
    and hence $n^2 +n$ is even.
  \end{proof}
  \qquad\\
  \refl Think of how you might prove this proposition without
  resorting to cases
\end{frame}

\begin{frame}
  \frametitle{Proof by cases: Inference rules}
  \begin{mathpar}
    \inferrule
        {
          (\phi_1\lor\cdots\lor\phi_n)\rightarrow \psi
        }
        {
          \phi_1\rightarrow \psi\\
          \cdots\\
          \phi_n\rightarrow \psi
        }
        \textsc{\ \ CaseIntro}

        \inferrule
            {
              \phi_1\rightarrow \psi\\
              \cdots\\
              \phi_n\rightarrow \psi
            }
            {
              (\phi_1\lor\cdots\lor\phi_n) \rightarrow \psi
            }
            \textsc{\ \ CaseElim}
  \end{mathpar}
  In short, $((\phi_1\lor\cdots\lor\phi_n) \rightarrow
  \psi)\equiv((\phi_1\rightarrow\psi)\land\cdots\land(\phi_n\rightarrow\psi))$.
\end{frame}

\begin{frame}
  \frametitle{Proof by cases}
  \begin{itemize}
  \item Pros
    \begin{itemize}
    \item Conceptually simple
    \item Well-chosen cases can make proofs easier to write
    \end{itemize}
    \qquad\\\qquad\\
  \item Cons
    \begin{itemize}
    \item Too many cases are fun for no one (especially readers)
    \item Considered inelegant
    \end{itemize}
    \qquad\\\qquad\\
  \end{itemize}
  \refl Come up with your own proposition and prove it by cases
\end{frame}

\begin{frame}
  \frametitle{Proof by contrapositive}

  \begin{itemize}
  \item Sometimes a direct proof of $P\rightarrow Q$ can be really
    annoying
  \item Proving the \textbf{contrapositive} $\neg Q\rightarrow \neg P$
    might be easier
  \item We can always opt to do this because
    \begin{mathpar}
      (P\rightarrow Q)\equiv (\neg Q \longrightarrow \neg P)
    \end{mathpar}
  \end{itemize}

  \exer What is the contrapositive of these claims?
  \begin{itemize}
  \item If today is Tuesday, then we have class today.
  \item If this is a cat, then it does not bark.
  \end{itemize}
  
\end{frame}

\begin{frame}
  \frametitle{Proof by contrapositive: Example}

  \begin{prop}
    If $r$ is an irrational number, then $\sqrt{r}$ is also an
    irrational number.
  \end{prop}
  \begin{proof}
    Prove the contrapositive: if $\sqrt{r}$ is not an irrational
    number, then $r$ is not an irrational number.\\

    We assume that $\sqrt{r}$ is a rational number. Then, there exists
    some integers $m$ and $n$ such that $\sqrt{r} =
    \frac{m}{n}$.\\

    Square both sides to get $\sqrt{r}^2 = \frac{m^2}{n^2}$. We know
    that a square of integer is an integer, so $m^2$ and $n^2$ are
    both integers. We get that $r = \frac{m^2}{n^2}$, which means that
    $r$ is a rational number and not an irrational number.
  \end{proof}
  \qquad\\ \refl How might you prove this directly? Is it easier or
  harder to prove this directly or with contrapositive?

\end{frame}


\end{document}
